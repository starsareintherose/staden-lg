\documentstyle[a4,11pt]{article}

\title{Installing the Staden Package}
\author{Simon Dear}
\date{21 May 1993}



\begin{document}
\maketitle



\section{Introduction}

On the accompanying tape you will find executables for 
one of SunOS 4.x, Sun
Solaris 2.x, DEC Ultrix, DEC OSF/1 and Silicon Graphics SGI operating systems.
Also there are sources for all the programs in the Staden package.
Programs in the package are:
\begin{description}

\item[mep and xmep] Motif exploration program.
\item[nip and xnip] Nucleotide interpretation program.
\item[nipl] Nucleotide interpretation program (library).
Searches nucleotide libraries for patterns of motifs.
\item[pip and xpip] Protein interpretation program.
\item[pipl] Protein interpretation program (library).
Searches protein libraries for patterns of motifs.
\item[sip and xsip] Similarity investigation program.
\item[sipl] Similarity investigation program (library).
Compares a probe protein or nucleic acid sequence against
a library of sequences.
\item[sap and xsap] The original sequence assembly program.
\item[bap and xbap] Our latest, most advanced sequence assembly program.
\item[dap and xdap] An obsolete assembly program, superceded by {\em bap}.
\item[lip] Library interface program.
\item[rep] Repeat examination program.
\item[ted] X windows utility for displaying and editing
fluorescent sequencing machine traces.
\item[splitp1, splitp2 and splitp3] Refer to help/SPLITP.MEM.
\item[sethelp] Builds online help files.
\item[gip] Gel input program.
\item[convert] Converts between {\em xdap\/} and {\em xbap\/} databases.
\item[cop and cop-bap] Checks completed {\em xdap\/} and {\em xbap\/}
databases for editing errors.
\item[trace2seq] Extracts sequence from trace files.
\item[getABISampleName] Extracts sample names from ABI trace files.
\item[makeSCF] Converts existing trace files to the compact
SCF format.
\item[alfsplit] Splits the Pharmacia A.L.F. gel
file into multiple files, one for each sample.
\item[frog] Relabels lanes in ABI trace files.
\item[+ numerous scripts (including {\em squirrel (v1.4)\/})]

\end{description}


\section{Requirements}

You will need a tape drive to read the software off the distribution
tape (QIC-150, TK50, or Exabyte). You will also need a large amount of
disk storage to accommodate the whole package. For release
version-1993.0, requirements were
31Mb (SunOS 4.x),
36Mb (Sun Solaris 2.x)
30Mb (DEC Ultrix)
37Mb (DEC OSF/1)
and
27Mb (Silicon Graphics SGI.)


To compile the Staden package you will require:
\begin{itemize}
\item An ANSI C compiler.
\item A FORTRAN-77 compiler.
\item X11 (Release 4 or 5).
\item GNU make (except with SunOS and Solaris 2.x.)
\end{itemize}

\section{Installation}

To install the package,
\begin{enumerate}
\item Create a directory for where you would like the software to be
placed. You may have to be superuser to do this.
      \begin{verbatim} mkdir /home/Staden\end{verbatim}
\item Change to this directory.
      \begin{verbatim} cd /home/Staden\end{verbatim}
\item Place the tape into the tape unit.
\item Extract the software off the distribution tape (NOTE: the device name may be
different on your machine):
      \begin{verbatim} tar xvf /dev/rst0\end{verbatim}
\item C shell users should set the environment variable {\bf STADENROOT}
to be the directory where the package is installed and source the file
{\em staden.login} found there. This is best done by adding lines to their
{\em .login} file:
\begin{verbatim}
    setenv STADENROOT /home/Staden
    source $STADENROOT/staden.login
\end{verbatim}
Users of the Bourne shell, sh, should similarly add lines their {\em .profile} file:
\begin{verbatim}  
    STADENROOT=/home/Staden
    export STADENROOT
    . $STADENROOT/staden.profile
\end{verbatim}

The startup routines set environment variables and modify the shell's
search path so that it can find the programs in the Staden Package.
When users next log on to the system, they will be able to use the
programs.

\end{enumerate}


\section {Installation on Unsupported Platforms}

Install the software as you would for a supported machine.  You will
need to remake all executables.  The script {\em Staden\_install} can
be used to help recompile the package. A large number of
assumptions have been made, and you may need to change the makefiles
to suit your system.

The sources have been organised into subdirectories of the directory
{\bf src}. In {\bf Misc} are routines common to many programs. They
should be made first.  In {\bf staden} are all the programs of the
Staden suite ({\em mep}, {\em nip}, {\em pip}, {\em sap}, {\em sip},
{\em dap}, {\em gip}, {\em vep}, {\em lip} and {\em rep}) with the
exception of {\em bap}.  Code for our latest sequence assembly program
{\em bap} is in directories {\bf bap} and {\bf bap/osp-bits}.  Make
the objects in {\bf staden} first, then the ones in {\bf
bap/osp-bits}, and finally the ones in {\bf bap}. In {\bf ted} is the
trace editing program.


\section {Other Software Provided}

Other software and scripts can be found in the {\bf alf\/}, {\bf
abi\/}, {\bf cop\/}, {\bf getMCH\/}, {\bf scf\/}, {\bf frog\/} and {\bf
scripts}
directories.
Each directory contains documentation describing the programs
contained.

Since release version-1993.0 we have distributed the {\em squirrel (v1.4)}
package. Please read the disclaimer that accompanies this software.
Additional sources and scripts can be found in {\bf expGetSeq}, {\bf vepe},
{\bf newted} and {\bf squirrel-1.4} directories.

Many scripts (including {\em squirrel}) and filters were developed at the MRC-LMB for
{\bf INTERNAL USE ONLY}.
We are aware that people elsewhere will want to develop
similar software.
We include them in the Staden Package merely as {\bf EXAMPLES} of
what has been achieved elsewhere.
{\bf THESE SCRIPTS WILL NOT WORK ON YOUR SYSTEM WITHOUT MODIFICATION.}

\section {When All Else Fails...}
If you have any problems please contact the authors,
\mbox{Rodger Staden}
\mbox{(\em rs@mrc-lmba.cam.ac.uk\/)},
\mbox{Simon Dear}
\mbox{(\em sd@mrc-lmba.cam.ac.uk\/)}
and
\mbox{James Bonfield}
\mbox{(\em jkb@mrc-lmba.cam.ac.uk\/)},
by email or by writing to us at:
MRC Laboratory of Molecular Biology, Hills Road, Cambridge, \mbox{CB2 2QH}, U.K.
We also welcome general comments on the package.

\end{document}
